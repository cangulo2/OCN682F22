% Options for packages loaded elsewhere
\PassOptionsToPackage{unicode}{hyperref}
\PassOptionsToPackage{hyphens}{url}
%
\documentclass[
]{article}
\usepackage{lmodern}
\usepackage{amssymb,amsmath}
\usepackage{ifxetex,ifluatex}
\ifnum 0\ifxetex 1\fi\ifluatex 1\fi=0 % if pdftex
  \usepackage[T1]{fontenc}
  \usepackage[utf8]{inputenc}
  \usepackage{textcomp} % provide euro and other symbols
\else % if luatex or xetex
  \usepackage{unicode-math}
  \defaultfontfeatures{Scale=MatchLowercase}
  \defaultfontfeatures[\rmfamily]{Ligatures=TeX,Scale=1}
\fi
% Use upquote if available, for straight quotes in verbatim environments
\IfFileExists{upquote.sty}{\usepackage{upquote}}{}
\IfFileExists{microtype.sty}{% use microtype if available
  \usepackage[]{microtype}
  \UseMicrotypeSet[protrusion]{basicmath} % disable protrusion for tt fonts
}{}
\makeatletter
\@ifundefined{KOMAClassName}{% if non-KOMA class
  \IfFileExists{parskip.sty}{%
    \usepackage{parskip}
  }{% else
    \setlength{\parindent}{0pt}
    \setlength{\parskip}{6pt plus 2pt minus 1pt}}
}{% if KOMA class
  \KOMAoptions{parskip=half}}
\makeatother
\usepackage{xcolor}
\IfFileExists{xurl.sty}{\usepackage{xurl}}{} % add URL line breaks if available
\IfFileExists{bookmark.sty}{\usepackage{bookmark}}{\usepackage{hyperref}}
\hypersetup{
  hidelinks,
  pdfcreator={LaTeX via pandoc}}
\urlstyle{same} % disable monospaced font for URLs
\usepackage[margin=1in]{geometry}
\usepackage{color}
\usepackage{fancyvrb}
\newcommand{\VerbBar}{|}
\newcommand{\VERB}{\Verb[commandchars=\\\{\}]}
\DefineVerbatimEnvironment{Highlighting}{Verbatim}{commandchars=\\\{\}}
% Add ',fontsize=\small' for more characters per line
\usepackage{framed}
\definecolor{shadecolor}{RGB}{248,248,248}
\newenvironment{Shaded}{\begin{snugshade}}{\end{snugshade}}
\newcommand{\AlertTok}[1]{\textcolor[rgb]{0.94,0.16,0.16}{#1}}
\newcommand{\AnnotationTok}[1]{\textcolor[rgb]{0.56,0.35,0.01}{\textbf{\textit{#1}}}}
\newcommand{\AttributeTok}[1]{\textcolor[rgb]{0.77,0.63,0.00}{#1}}
\newcommand{\BaseNTok}[1]{\textcolor[rgb]{0.00,0.00,0.81}{#1}}
\newcommand{\BuiltInTok}[1]{#1}
\newcommand{\CharTok}[1]{\textcolor[rgb]{0.31,0.60,0.02}{#1}}
\newcommand{\CommentTok}[1]{\textcolor[rgb]{0.56,0.35,0.01}{\textit{#1}}}
\newcommand{\CommentVarTok}[1]{\textcolor[rgb]{0.56,0.35,0.01}{\textbf{\textit{#1}}}}
\newcommand{\ConstantTok}[1]{\textcolor[rgb]{0.00,0.00,0.00}{#1}}
\newcommand{\ControlFlowTok}[1]{\textcolor[rgb]{0.13,0.29,0.53}{\textbf{#1}}}
\newcommand{\DataTypeTok}[1]{\textcolor[rgb]{0.13,0.29,0.53}{#1}}
\newcommand{\DecValTok}[1]{\textcolor[rgb]{0.00,0.00,0.81}{#1}}
\newcommand{\DocumentationTok}[1]{\textcolor[rgb]{0.56,0.35,0.01}{\textbf{\textit{#1}}}}
\newcommand{\ErrorTok}[1]{\textcolor[rgb]{0.64,0.00,0.00}{\textbf{#1}}}
\newcommand{\ExtensionTok}[1]{#1}
\newcommand{\FloatTok}[1]{\textcolor[rgb]{0.00,0.00,0.81}{#1}}
\newcommand{\FunctionTok}[1]{\textcolor[rgb]{0.00,0.00,0.00}{#1}}
\newcommand{\ImportTok}[1]{#1}
\newcommand{\InformationTok}[1]{\textcolor[rgb]{0.56,0.35,0.01}{\textbf{\textit{#1}}}}
\newcommand{\KeywordTok}[1]{\textcolor[rgb]{0.13,0.29,0.53}{\textbf{#1}}}
\newcommand{\NormalTok}[1]{#1}
\newcommand{\OperatorTok}[1]{\textcolor[rgb]{0.81,0.36,0.00}{\textbf{#1}}}
\newcommand{\OtherTok}[1]{\textcolor[rgb]{0.56,0.35,0.01}{#1}}
\newcommand{\PreprocessorTok}[1]{\textcolor[rgb]{0.56,0.35,0.01}{\textit{#1}}}
\newcommand{\RegionMarkerTok}[1]{#1}
\newcommand{\SpecialCharTok}[1]{\textcolor[rgb]{0.00,0.00,0.00}{#1}}
\newcommand{\SpecialStringTok}[1]{\textcolor[rgb]{0.31,0.60,0.02}{#1}}
\newcommand{\StringTok}[1]{\textcolor[rgb]{0.31,0.60,0.02}{#1}}
\newcommand{\VariableTok}[1]{\textcolor[rgb]{0.00,0.00,0.00}{#1}}
\newcommand{\VerbatimStringTok}[1]{\textcolor[rgb]{0.31,0.60,0.02}{#1}}
\newcommand{\WarningTok}[1]{\textcolor[rgb]{0.56,0.35,0.01}{\textbf{\textit{#1}}}}
\usepackage{longtable,booktabs}
% Correct order of tables after \paragraph or \subparagraph
\usepackage{etoolbox}
\makeatletter
\patchcmd\longtable{\par}{\if@noskipsec\mbox{}\fi\par}{}{}
\makeatother
% Allow footnotes in longtable head/foot
\IfFileExists{footnotehyper.sty}{\usepackage{footnotehyper}}{\usepackage{footnote}}
\makesavenoteenv{longtable}
\usepackage{graphicx,grffile}
\makeatletter
\def\maxwidth{\ifdim\Gin@nat@width>\linewidth\linewidth\else\Gin@nat@width\fi}
\def\maxheight{\ifdim\Gin@nat@height>\textheight\textheight\else\Gin@nat@height\fi}
\makeatother
% Scale images if necessary, so that they will not overflow the page
% margins by default, and it is still possible to overwrite the defaults
% using explicit options in \includegraphics[width, height, ...]{}
\setkeys{Gin}{width=\maxwidth,height=\maxheight,keepaspectratio}
% Set default figure placement to htbp
\makeatletter
\def\fps@figure{htbp}
\makeatother
\setlength{\emergencystretch}{3em} % prevent overfull lines
\providecommand{\tightlist}{%
  \setlength{\itemsep}{0pt}\setlength{\parskip}{0pt}}
\setcounter{secnumdepth}{-\maxdimen} % remove section numbering

\author{}
\date{\vspace{-2.5em}}

\begin{document}

\hypertarget{question-0}{%
\subsubsection{Question 0}\label{question-0}}

\begin{itemize}
\tightlist
\item
  Draw a concepts diagram that uses all the following Github terminology

  \begin{itemize}
  \tightlist
  \item
    Include any missing keywords that will simplify the concepts
    diagrams
  \end{itemize}
\end{itemize}

\texttt{Push}, \texttt{Repository}, \texttt{Clone}, \texttt{Pull},
\texttt{Pull\ Request}, \texttt{Branch}, \texttt{Merging},
\texttt{Github\ Client}, \texttt{README\ file},
\texttt{Private\ or\ Public}

\begin{Shaded}
\begin{Highlighting}[]
\CommentTok{# Add your photo here}
\end{Highlighting}
\end{Shaded}

\hypertarget{question-1}{%
\paragraph{Question 1}\label{question-1}}

\texttt{matrix(c(1,2,3,4,5,6)\ ,\ nrow\ =\ 3)}

\begin{itemize}
\tightlist
\item
  Running the expression produces the following matrix
\end{itemize}

\begin{longtable}[]{@{}ll@{}}
\toprule
\endhead
1 & 4\tabularnewline
2 & 5\tabularnewline
3 & 6\tabularnewline
\bottomrule
\end{longtable}

\begin{itemize}
\tightlist
\item
  How can you modify the call to \texttt{matrix()} to produce the
  following matrix instead?
\end{itemize}

\begin{longtable}[]{@{}lll@{}}
\toprule
& col\_1 & col\_2\tabularnewline
\midrule
\endhead
row\_1 & 1 & 2\tabularnewline
row\_2 & 3 & 4\tabularnewline
row\_3 & 5 & 6\tabularnewline
\bottomrule
\end{longtable}

\begin{itemize}
\tightlist
\item
  Note that you need to name the columns (col\_1 and col\_2) and name
  the rows (row\_1, row\_2, row\_3)
\end{itemize}

Hint: Use the \texttt{?} symbol to invoke the matrix documentation

\begin{Shaded}
\begin{Highlighting}[]
\CommentTok{# Write your answer here}
\NormalTok{test_matrix <-}\StringTok{ }\KeywordTok{matrix}\NormalTok{(}\KeywordTok{c}\NormalTok{(}\DecValTok{1}\OperatorTok{:}\DecValTok{6}\NormalTok{), }\DataTypeTok{nrow=}\DecValTok{3}\NormalTok{)}
\KeywordTok{rownames}\NormalTok{(test_matrix)<-}\StringTok{ }\KeywordTok{c}\NormalTok{(}\StringTok{"row_1"}\NormalTok{, }\StringTok{"row_2"}\NormalTok{, }\StringTok{"row_3"}\NormalTok{)}
\KeywordTok{colnames}\NormalTok{(test_matrix)<-}\StringTok{ }\KeywordTok{c}\NormalTok{(}\StringTok{"col_1"}\NormalTok{, }\StringTok{"col_2"}\NormalTok{)}
\end{Highlighting}
\end{Shaded}

\hypertarget{question-2}{%
\paragraph{Question 2}\label{question-2}}

\begin{itemize}
\tightlist
\item
  Load then sort the airquality data frame on its \texttt{Temp} and
  \texttt{Solar.R} columns in reverse order (largest to smallest values)

  \begin{itemize}
  \tightlist
  \item
    The function to sort a data frame is called order
  \end{itemize}
\item
  Display only the first 15 lines of your table
\end{itemize}

\begin{Shaded}
\begin{Highlighting}[]
\CommentTok{# Write your answer here}
\NormalTok{airqual_temp<-airquality[}\KeywordTok{order}\NormalTok{(airquality}\OperatorTok{$}\NormalTok{Temp, }\DataTypeTok{decreasing =} \OtherTok{TRUE}\NormalTok{),]}
\NormalTok{airqual_solar<-airquality[}\KeywordTok{order}\NormalTok{(airquality}\OperatorTok{$}\NormalTok{Solar.R, }\DataTypeTok{decreasing =} \OtherTok{TRUE}\NormalTok{),]}
\KeywordTok{head}\NormalTok{(airqual_temp, }\DataTypeTok{n =} \DecValTok{15}\NormalTok{)}
\end{Highlighting}
\end{Shaded}

\begin{verbatim}
##     Ozone Solar.R Wind Temp Month Day
## 120    76     203  9.7   97     8  28
## 122    84     237  6.3   96     8  30
## 121   118     225  2.3   94     8  29
## 123    85     188  6.3   94     8  31
## 42     NA     259 10.9   93     6  11
## 126    73     183  2.8   93     9   3
## 127    91     189  4.6   93     9   4
## 43     NA     250  9.2   92     6  12
## 69     97     267  6.3   92     7   8
## 70     97     272  5.7   92     7   9
## 102    NA     222  8.6   92     8  10
## 125    78     197  5.1   92     9   2
## 75     NA     291 14.9   91     7  14
## 124    96     167  6.9   91     9   1
## 40     71     291 13.8   90     6   9
\end{verbatim}

\begin{Shaded}
\begin{Highlighting}[]
\KeywordTok{head}\NormalTok{(airqual_solar, }\DataTypeTok{n =} \DecValTok{15}\NormalTok{)}
\end{Highlighting}
\end{Shaded}

\begin{verbatim}
##    Ozone Solar.R Wind Temp Month Day
## 16    14     334 11.5   64     5  16
## 45    NA     332 13.8   80     6  14
## 41    39     323 11.5   87     6  10
## 19    30     322 11.5   68     5  19
## 46    NA     322 11.5   79     6  15
## 22    11     320 16.6   73     5  22
## 67    40     314 10.9   83     7   6
## 4     18     313 11.5   62     5   4
## 17    34     307 12.0   66     5  17
## 7     23     299  8.6   65     5   7
## 84    NA     295 11.5   82     7  23
## 85    80     294  8.6   86     7  24
## 40    71     291 13.8   90     6   9
## 75    NA     291 14.9   91     7  14
## 13    11     290  9.2   66     5  13
\end{verbatim}

\hypertarget{question-3}{%
\paragraph{Question 3}\label{question-3}}

\begin{itemize}
\tightlist
\item
  Sort the airquality data frame on its \texttt{Temp} in decreasing
  order and \texttt{Solar.R} in increasing order
\item
  Display only the first 15 lines of your table
\end{itemize}

\begin{Shaded}
\begin{Highlighting}[]
\CommentTok{# Write your answer here}
\NormalTok{airqual_temp1<-airquality[}\KeywordTok{order}\NormalTok{(airquality}\OperatorTok{$}\NormalTok{Temp, }\DataTypeTok{decreasing =} \OtherTok{TRUE}\NormalTok{),]}
\NormalTok{airqual_solar1<-airquality[}\KeywordTok{order}\NormalTok{(airquality}\OperatorTok{$}\NormalTok{Solar.R, }\DataTypeTok{decreasing =} \OtherTok{FALSE}\NormalTok{),]}
\end{Highlighting}
\end{Shaded}

\hypertarget{question-4}{%
\paragraph{Question 4}\label{question-4}}

\begin{itemize}
\item
  There are various ways to select a subset of observations from a data
  frame.
\item
  Consult your
  \href{https://cran.r-project.org/doc/contrib/Baggott-refcard-v2.pdf}{R
  Reference Card}, see \texttt{Data\ Selection\ and\ Manipulation}
  section.

  \begin{itemize}
  \tightlist
  \item
    What operations can you use to select all observations where the
    temperature is 72. Give at least two different answers to this
    question
  \end{itemize}
\end{itemize}

\begin{Shaded}
\begin{Highlighting}[]
\CommentTok{# Write your answer here}
\KeywordTok{subset}\NormalTok{(airquality, Temp }\OperatorTok{==}\StringTok{ }\DecValTok{72}\NormalTok{) }\CommentTok{#answer 1}
\end{Highlighting}
\end{Shaded}

\begin{verbatim}
##     Ozone Solar.R Wind Temp Month Day
## 2      36     118  8.0   72     5   2
## 48     37     284 20.7   72     6  17
## 114     9      36 14.3   72     8  22
\end{verbatim}

\begin{Shaded}
\begin{Highlighting}[]
\NormalTok{temp_}\DecValTok{72}\NormalTok{ <-}\StringTok{ }\NormalTok{airquality}\OperatorTok{$}\NormalTok{temp }\OperatorTok{==}\StringTok{ }\DecValTok{72} \CommentTok{#answer 2}
\NormalTok{airquality[temp_}\DecValTok{72}\NormalTok{,]}
\end{Highlighting}
\end{Shaded}

\begin{verbatim}
## [1] Ozone   Solar.R Wind    Temp    Month   Day    
## <0 rows> (or 0-length row.names)
\end{verbatim}

\hypertarget{question-6}{%
\paragraph{Question 6}\label{question-6}}

\begin{itemize}
\tightlist
\item
  You may have noticed when working with the \texttt{airqulity} data
  that some values show as \texttt{NA}
\item
  \texttt{NA} stands for not available, or missing values.
\item
  A major part of data wrangling consists of cleaning missing values by
  either:

  \begin{itemize}
  \tightlist
  \item
    Dropping the lines that have missing values
  \item
    Sometimes we can drop the column with missing values if the column
    is made of predominantly missing values
  \item
    Imputing the missing values, which uses educated guesses (or more
    complex algorithms) to fill the missing values
  \end{itemize}
\item
  Find and remove all rows that are missing values for the
  \texttt{Solar.R} or \texttt{Ozone} variables
\item
  Save the cleaned data to a new data frame called airquality\_no\_na

  \begin{itemize}
  \tightlist
  \item
    How many lines have been removed?
  \end{itemize}
\end{itemize}

\begin{Shaded}
\begin{Highlighting}[]
\CommentTok{# Write your answer here}

\NormalTok{air_quality_no_na<-airquality[}\KeywordTok{is.na}\NormalTok{(airquality}\OperatorTok{$}\NormalTok{Solar.R) }\OperatorTok{==}\StringTok{ }\OtherTok{FALSE} \OperatorTok{&}\StringTok{ }\KeywordTok{is.na}\NormalTok{(airquality}\OperatorTok{$}\NormalTok{Ozone) }\OperatorTok{==}\StringTok{ }\OtherTok{FALSE}\NormalTok{,]}

\CommentTok{# 42 lines were removed}
\end{Highlighting}
\end{Shaded}

\hypertarget{question-7}{%
\paragraph{Question 7}\label{question-7}}

\begin{itemize}
\tightlist
\item
  Let's use a different strategy and impute the missing value.

  \begin{itemize}
  \tightlist
  \item
    replace the missing values for Solar.R using that month's average.
  \item
    Example:

    \begin{itemize}
    \tightlist
    \item
      The missing value for line 6 should be replaced with the average
      for month 5.
    \item
      The missing value for line 97 should be replaced with the average
      for month 8.
    \end{itemize}
  \end{itemize}
\end{itemize}

\begin{Shaded}
\begin{Highlighting}[]
\CommentTok{# Write your answer here}
\NormalTok{na_solar<-airquality[}\KeywordTok{is.na}\NormalTok{(airquality}\OperatorTok{$}\NormalTok{Solar.R),] }\CommentTok{#subset rows where solar.r is NA}
\NormalTok{mean_}\DecValTok{5}\NormalTok{ <-}\StringTok{ }\KeywordTok{mean}\NormalTok{(airquality}\OperatorTok{$}\NormalTok{Solar.R[airquality}\OperatorTok{$}\NormalTok{Month }\OperatorTok\StringTok{ }\KeywordTok{unique}\NormalTok{(na_solar}\OperatorTok{$}\NormalTok{Month)[}\DecValTok{1}\NormalTok{]], }\DataTypeTok{na.rm =} \OtherTok{TRUE}\NormalTok{) }\CommentTok{#finds mean of month 5}
\NormalTok{mean_}\DecValTok{8}\NormalTok{ <-}\StringTok{ }\KeywordTok{mean}\NormalTok{(airquality}\OperatorTok{$}\NormalTok{Solar.R[airquality}\OperatorTok{$}\NormalTok{Month }\OperatorTok\StringTok{ }\KeywordTok{unique}\NormalTok{(na_solar}\OperatorTok{$}\NormalTok{Month)[}\DecValTok{2}\NormalTok{]], }\DataTypeTok{na.rm =} \OtherTok{TRUE}\NormalTok{) }\CommentTok{#finds mean of month 8}
\NormalTok{copy_airquality <-}\StringTok{ }\NormalTok{airquality }\CommentTok{#because I don't want to mess with the original dataset }
\NormalTok{copy_airquality[}\KeywordTok{is.na}\NormalTok{(airquality}\OperatorTok{$}\NormalTok{Solar.R[airquality}\OperatorTok{$}\NormalTok{Month }\OperatorTok{==}\DecValTok{5}\NormalTok{]), }\DecValTok{2}\NormalTok{] <-}\StringTok{ }\NormalTok{mean_}\DecValTok{5} \CommentTok{#replaces month 5 na values with mean}
\NormalTok{copy_airquality[}\KeywordTok{is.na}\NormalTok{(airquality}\OperatorTok{$}\NormalTok{Solar.R[airquality}\OperatorTok{$}\NormalTok{Month }\OperatorTok{==}\DecValTok{8}\NormalTok{]), }\DecValTok{2}\NormalTok{] <-}\StringTok{ }\NormalTok{mean_}\DecValTok{8} \CommentTok{#replaces month 8 na values with mean}
\end{Highlighting}
\end{Shaded}

\end{document}
